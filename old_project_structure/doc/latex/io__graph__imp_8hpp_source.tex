\hypertarget{io__graph__imp_8hpp_source}{
\section{io\_\-graph\_\-imp.hpp}
}


\begin{footnotesize}\begin{alltt}
00001 \textcolor{preprocessor}{#ifndef HH\_IO\_GRAPH\_IMP\_HH}
00002 \textcolor{preprocessor}{}\textcolor{preprocessor}{#define HH\_IO\_GRAPH\_IMP\_HH}
00003 \textcolor{preprocessor}{}
00004 \textcolor{preprocessor}{#include<tuple>}
00005 \textcolor{preprocessor}{#include<boost/graph/adjacency\_list.hpp>}
00006 \textcolor{preprocessor}{#include"edge\_property.hpp"}
00007 \textcolor{preprocessor}{#include<string>}
00008 \textcolor{preprocessor}{#include<iostream>}
00009 \textcolor{preprocessor}{#include<sstream>}
00010 \textcolor{preprocessor}{#include<string>}
00011 \textcolor{preprocessor}{#include<fstream>}
00012 \textcolor{preprocessor}{#include<vector>}
00013 \textcolor{preprocessor}{#include<set>}
00014 \textcolor{preprocessor}{#include<utility>}
00015 
00016 \textcolor{preprocessor}{#include"\hyperlink{generic__point_8hpp}{generic_point.hpp}"}
00017 \textcolor{preprocessor}{#include"edge\_property.hpp"}
00018 \textcolor{preprocessor}{#include"io\_graph.hpp"}
00019 
00020 \textcolor{keyword}{template}<\textcolor{keyword}{typename} Graph, \textcolor{keyword}{typename} Po\textcolor{keywordtype}{int}>
00021 \textcolor{keywordtype}{void} initialize\_graph(\textcolor{keyword}{const} \textcolor{keywordtype}{int} src, \textcolor{keyword}{const} \textcolor{keywordtype}{int} tgt, Graph & G, \textcolor{keywordtype}{double} diam, \textcolor{keywordtype}{doubl
      e} length, Point \textcolor{keyword}{const} & SRC, Point \textcolor{keyword}{const} & TGT)\{
00022         
00023         \textcolor{keyword}{typedef} \textcolor{keyword}{typename} boost::graph\_traits<Graph>::edge\_descriptor edge\_descrip
      tor;
00024         
00025         std::set<int> vertex\_set;               \textcolor{comment}{// it's a set in order to avoid m
      ultiple additions of the same vertex}
00026         \textcolor{keywordtype}{bool} inserted;                          \textcolor{comment}{//if FALSE the vertex I'm trying 
      to add was already in the set }
00027 
00028         std::pair<std::set<int>::iterator, \textcolor{keywordtype}{bool}> set\_inserter;
00029         std::set<int>::iterator it;
00030         
00031         edge\_descriptor e;
00032         \textcolor{keywordtype}{bool} edge\_inserted;     
00033         
00035         std::tie(e, edge\_inserted) = boost::add\_edge(src, tgt, G);
00036         
00037         G[e].capacity = diam;
00038 
00039         G[e].length = length;
00040         
00041         \textcolor{comment}{// Inserting property of the source vertex, if the vertex wasn't inserted
       before}
00042         set\_inserter = vertex\_set.insert(src);
00043         \textcolor{keywordflow}{if}(set\_inserter.second)
00044                 G[src]=SRC;
00045 
00046         \textcolor{comment}{// Inserting property of the target vertex, if the vertex wasn't inserted
       before         }
00047         set\_inserter = vertex\_set.insert(tgt);
00048         \textcolor{keywordflow}{if}(set\_inserter.second)
00049                 G[tgt]=TGT;
00050 \}       \textcolor{comment}{//initialize\_graph}
00051         
00052                 
00053 \textcolor{keyword}{template}<\textcolor{keyword}{typename} Graph, \textcolor{keyword}{typename} Po\textcolor{keywordtype}{int}>
00054 \textcolor{keywordtype}{void} read\_zunino\_old\_format(Graph & G, std::string file\_name)\{ 
00055 
00056   \textcolor{comment}{/* I want to read from a file where data }
00057 \textcolor{comment}{     is written as}
00058 \textcolor{comment}{     }
00059 \textcolor{comment}{     line1: description}
00060 \textcolor{comment}{     line2: description}
00061 \textcolor{comment}{     from line 3: no - source\_v - target\_v - diameter - length - source\_coord - t
      arget\_coord}
00062 \textcolor{comment}{  */}
00063   
00064   std::ifstream file(file\_name.c\_str());
00065 
00066 
00067   Point SRC,TGT;                                
00068   \textcolor{keywordtype}{int} edge\_num, src, tgt;                       \textcolor{comment}{// they will read the first 3 num
      bers of each line }
00069   \textcolor{keywordtype}{double} diam, length;                          \textcolor{comment}{// they will read the remaining 8
      }
00070 
00071   \textcolor{comment}{// ignore the first two lines of the file}
00072   std::string dummyLine;
00073   std::getline(file, dummyLine);
00074   std::getline(file, dummyLine);
00075   
00076   \textcolor{comment}{// Until I reach end of file}
00077   \textcolor{keywordflow}{while} (!file.fail() && !file.eof())\{
00078     std::string s;
00079     std::getline(file,s);                                                                               \textcolor{comment}{/
      / read the the whole line}
00080     \textcolor{keywordflow}{if}(s.empty()) \textcolor{keywordflow}{continue};                                                                     \textcolor{comment}{/
      / empty line}
00081     std::istringstream tmp(s);                                                                  \textcolor{comment}{/
      / build an input sstream.}
00082     tmp>>edge\_num>>src>>tgt>>diam>>length>>SRC>>TGT;                     \textcolor{comment}{// read 
      from the input stream}
00083     \textcolor{keywordflow}{if}(!tmp.fail())\{
00085         initialize\_graph<Graph, point<3> >(src, tgt, G, diam, length, SRC, TGT);
00086     \}
00087   \}     \textcolor{comment}{// while}
00088   
00089   \textcolor{comment}{//Eliminiamo il vertice zero perché nei file di input non c'è, si parte dal v
      ertice 1:}
00090   \textcolor{comment}{//remove\_vertex(0, G);}
00091 
00092   \textcolor{comment}{//Plot dei vertici:}
00093   \textcolor{keyword}{typename} boost::graph\_traits<Graph>::vertex\_iterator vb, ve;
00094   \textcolor{keywordflow}{for}(tie(vb,ve) = vertices(G); vb != ve; ++vb)
00095         std::cout << *vb << std::endl;
00096   
00097   \textcolor{comment}{//Plot degli archi:}
00098   \textcolor{keyword}{typename} boost::graph\_traits<Graph>::edge\_iterator ebegin, eend;
00099   \textcolor{keywordflow}{for}( tie(ebegin, eend) = edges(G); ebegin != eend; ebegin++)
00100         std::cout << *ebegin << std::endl;
00101 \}       \textcolor{comment}{//read\_zunino\_old\_format}
00102 \textcolor{preprocessor}{#endif}
\end{alltt}\end{footnotesize}
