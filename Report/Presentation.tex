\documentclass[11pt]{beamer}
\usepackage[utf8]{inputenc}
%\usepackage[]{babel}
\usepackage{amsmath}
\usepackage{amsfonts}
\usepackage{amssymb}
\usepackage{graphicx}
\usetheme{Boadilla}

\AtBeginSection[]{
	\begin{frame}
		\vfill
		\centering
		\begin{beamercolorbox}[sep=8pt,center,shadow=true,rounded=true]{title}
			\usebeamerfont{title}\insertsectionhead\par%
		\end{beamercolorbox}
		\vfill
	\end{frame}
}

\author{Speranza Ilaria (matr. 854196) \\ Tantardini Mattia (matr. 858603)}
\title{BGLgeom Library}
%\subtitle{}
%\logo{}
\institute{\textbf{Politecnico di Milano}}
\date{March 3, 2017}
\subject{Advanced Programming for Scientific Computing}
%\setbeamercovered{transparent}
%\setbeamertemplate{navigation symbols}{}

\begin{document}
	\begin{frame}
		\maketitle
	\end{frame}

	\begin{frame}
		\frametitle{Project's objectives}
		\begin{itemize}
			\item Add geometric features to a BGL graph 
			\item Implement linear, b-spline and "generic" edges
			\item Provide mesh generator on each edge
			\item Handle I/O operations from/to useful formats (.pts, .vtp, ASCII) 
			\item Fracture\_intersection and Network\_diffusion applications 
		\end{itemize}
	\end{frame}

	\section{The library}
		\begin{frame}
			\frametitle{Briefs on Boost Graph Library}
			\begin{block}{Adjacency\_list}
				\texttt{adjacency\_list< OutEdgeList, VertexList, Directed, VertexProperties, EdgeProperties >} \newline
				Template class representing the graph with a two dimensional structure: 
				a \texttt{VertexList}, containing all the vertices, and an \texttt{OutEdgeList} associated to each vertex, containing all its out-edges.
			\end{block}
		
			\begin{block}{Vertex \& Edge descriptors and iterators}
				Descriptors are the types for vertices and edges representative objects; iterators allow to traverse graph's vertex and edge sets.
			\end{block}
		
			\begin{block}{Bundled properties}
				Structs associated to each vertex and each edge containing their properties and methods
			\end{block}	
		\end{frame}
	
		\begin{frame}
			\frametitle{BGLgeom}
			This library has been developed to provide an environment where building and running all those applications which have both a graph topological structure and a geometric description for vertices (position in the space) and edges (which could not be linear).\\
			BGLgeom implements also some input/output utilities, to make this library 'compatible' with other softwares.
		\end{frame}
	
		\begin{frame}{Inside BGLgeom}
			\begin{itemize}
				\item \textbf{Adapters for BGL}: layers and additional functions to hidden the most used native BGL ones and to improve readibility and ease of use.
				\item \textbf{Classes to build graph properties}: the main part of the library; classes to be associated vertices and edges
				as properties including the basic geometric requirements.
				\item \textbf{Geometrical and numerical utilities}: Code to compute integrals, generate meshes, compute intersections between linear edges.
				\item \textbf{I/O utilities}: one reader class to read tabular ASCII files, and three writer classes to produce three different types of output: ASCII, .pts and .vtp files.
				\item \textbf{Tests}: source code examples to show how the main classes and writers work.
			\end{itemize}
		\end{frame}
\end{document}